\documentclass[brazil]{beamer}
\usepackage{beamerthemesplit}
\usepackage[brazilian]{babel}
\usepackage[utf8]{inputenc}
\usepackage{hyperref}
\usepackage{color}
\usepackage{xcolor}
\usepackage{amssymb}
\usepackage{amsmath}
\usepackage{fancybox}
\usepackage{ulem}
\usepackage{listings}
\usetheme{Szeged}
\usecolortheme{seagull}
\usefonttheme{structuresmallcapsserif}

% lembrete: \vspace{0.4cm}

\title{Can We Pay For What We Get In 3G Data Access?}
\author{Diogo Haruki Kykuta}
\date{}

\begin{document}

\frame{\titlepage}

%-------------------------------------
% INTRODUÇÃO
%-------------------------------------
\section{Introdução}


\frame{\begin{center}
        \Huge Introdução
\end{center}}

\subsection{Motivação}

\begin{frame}[fragile]
    \frametitle{Motivação}
    \begin{itemize}
        \item Experiências ruins com 3G
        \item Queda de velocidade no final do mês, mesmo eu achando que estava abaixo do "limite"
    \end{itemize}
\end{frame}

\begin{frame}[fragile]
    De acordo com a operadora, posso navegar até 300MB no mês usando até 1Mbps, após isso, a velocidade cai
    para 50 kbps.

    \pause
    \vspace{0.4cm}
    Mas eu já experimentei essa diminuição de velocidade mesmo não tendo atingido o limite.
\end{frame}

\subsection{O paper}
\begin{frame}[fragile]
    \frametitle{O paper}
    \begin{center}
        Chunyi Peng, Guan-Hua Tu, Chi-Yu Li, Songwu Lu
    \end{center}
    \vspace{0.4cm}
    \begin{itemize}
        \item Estudo em casos normais e extremos.
        \item UDP e TCP.
        \item Navegar sem serem cobrados.

    \end{itemize}
\end{frame}

\subsection{Os testes}
\begin{frame}[fragile]
    Os testes foram feitos usando duas grandes operadoras dos EUA. \\
    Com algumas comparações (em certos casos) com uma terceira operadora e com operadoras chinesas.
\end{frame}

\section{Casos Extremos}
\frame{\begin{center}
        \Huge Casos Extremos
\end{center}}

\subsection{UDP sem controle}

\begin{frame}[fragile]
    UDP não tem verificação de recebimento de pacote. \\
    A operadora recebe os pacotes do servidor, mas não consegue mandar para o dispositivo móvel. Mas ainda assim, são cobrados.
\end{frame}

\begin{frame}[fragile]
    O que importa é a quantidade de pacotes que chega na operadora, e não quantos pacotes chegam na Unidade Móvel.
\end{frame}

\subsection{Velocidade maior do que a compatível}

\begin{frame}[fragile]
    TCP tem controle de fluxo, portanto, pacotes não recebidos não devolvem ACK. Assim, não são tantos pacotes perdidos e a diferença não é tão grande.
    
\end{frame}

\begin{frame}[fragile]
    UDP não tem. Portanto os pacotes entram na operadora com a velocidade enviada pelo servidor e são armazenadas em um buffer para serem enviadas conforme for possível. 

    \vspace{0.3cm}
    Mas esse buffer tem tamanho limitado $\rightarrow$ perda de pacotes.
\end{frame}

\section{Casos Normais}

\frame{\begin{center}
        \Huge Casos Normais
\end{center}}

\subsection{Envio de dados para IP errado}

\begin{frame}[fragile]
    Enviar dados via UDP para um IP inválido gera cobrança.
\end{frame}

\subsection{Links quebrados}

\begin{frame}[fragile]
    \frametitle{Página não encontrada}
    Página não encontrada
\end{frame}

\begin{frame}[fragile]
    \frametitle{Página com 50 links de imagens inexistentes}
    Teste
\end{frame}

\begin{frame}[fragile]
\end{frame}

\begin{frame}[fragile]
\end{frame}



\end{document}
